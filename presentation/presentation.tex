%% Beispiel-Präsentation mit LaTeX Beamer im KIT-Design
%% entsprechend den Gestaltungsrichtlinien vom 1. August 2020
%%
%% Siehe https://sdqweb.ipd.kit.edu/wiki/Dokumentvorlagen

%% Beispiel-Präsentation
\documentclass[en]{sdqbeamer} 
 
%% Titelbild
\titleimage{../images/title_header_5}

%% Gruppenlogo
\grouplogo{teco_logo.png} 

%% Gruppenname und Breite (Standard: 50 mm)
\groupname{
Chair of Pervasive Computing Systems/TECO\\
Institute of Telematics, Department of Informatics\\
}
%\groupnamewidth{50mm}

% Beginn der Präsentation

\title[Ear-Based Temperature Probing]{Ear-Based Temperature Probing: \\ Sensor Placement and Fusion for Wearable Applications}
\subtitle{Chair of Pervasive Computing Systems / TECO} 
\author[David Laubenstein]{David Laubenstein, Supervisor: Tobias Röddiger}

\date[11/08/2023]{Nov 08, 2023}

% Literatur 
 
\usepackage[bibstyle=numeric,hyperref,backend=biber]{biblatex}
\usepackage[inkscapeformat=png]{svg}
\addbibresource{../thesis-doc/Literature.bib}
\bibhang1em

\begin{document}
 
%Titelseite
\KITtitleframe

%Inhaltsverzeichnis
% sollte weg laut betreuern
% \begin{frame}{Inhaltsverzeichnis}
% \tableofcontents
% \end{frame}

\section{Motivation}
\begin{frame}{Motivation}
    \begin{itemize}
        \item current state of temperature measurement \cite{TemperatureDigitalGlassa}
    \end{itemize}
    \begin{center}
        \includegraphics[scale=0.16]{proposal-presentation/images/thermometer_types.jpg}    
    \end{center}
\end{frame}

\begin{frame}{Motivation}
    \begin{center}
        \includegraphics[scale=0.16]{proposal-presentation/images/inears/earbuds_picture.jpg}
    \end{center}
\end{frame}

% \section{Problem}
% \begin{frame}{Problem}
%     \begin{itemize}
%         \item accuracy and reliability
%         \begin{itemize}
%             \item sensor location
%             \item skin contact
%             \item calibration
%         \end{itemize}
%         \item tympanic membrane measurement set up
%         \begin{itemize}
%             \item alignment
%             \item ear wax
%         \end{itemize}
%     \end{itemize}
% \end{frame}

% \begin{frame}[fragile]{Question}
%     \begin{column}{0.50\textwidth}
%         \begin{itemize}
%             \item measure temperature at different locations in and around the ears
%             \item hypotheses
%             \begin{itemize}
%                 \item correlations with activity
%                 \item quality can be improved through IMU
%                 \item combination of signals improves the quality
%                 \item skin temperature around the ear strongly correlates with core body temperature.
%             \end{itemize}
%         \end{itemize}
%     \end{column}
%     \begin{column}{0.45\textwidth}
%         \includegraphics[scale=0.26]{proposal-presentation/images/open_earable_new.png}
%     \end{column}
% \end{frame}

% \begin{frame}[fragile]{Related Work}
% \begin{column}{0.47\textwidth}
%         \begin{itemize}
%         \item huge research field for ...
%         \begin{itemize}
%             \item temperature measurement on the tympanic membrane
%             \item body temperature measurement
%         \end{itemize}
%         \item no studies on other positions in/around the ear
%     \end{itemize}
%     \vspace{1cm}
%     \includegraphics[scale=0.26]{proposal-presentation/images/relatedWork/relatedWork3.png}
%     \end{column}
%     \begin{column}{0.43\textwidth}
%     \vspace{-0.4cm}
%         \includegraphics[scale=0.19]{proposal-presentation/images/relatedWork/relatedWorkMerged2.png}
%     \end{column}
% \end{frame}

\section{Planned Approach}
\begin{frame}{Planned Approach}
    \begin{itemize}
        \item device to measure ear-based temperature data
        \begin{itemize}
            \item OpenEarable platform adaption
            \item MLX temperature sensor
        \end{itemize}
        \item 2 studies (baseline surveys and environmental influences, under stress conditions)
    \end{itemize}
    \begin{center}
        \includegraphics[scale=0.17]{../thesis-doc/images/ear_measurement_points/emp.png}    
    \end{center}
\end{frame}

% \section{Expected Result}
% \begin{frame}{Expected Result}
%     \begin{itemize}
%         \item insights into sensor placement in/around the ear
%         \item combination of sensors to improve data quality
%         \item support the development of more accurate and reliable wearable devices for biometric applications
%     \end{itemize}
% \end{frame}

\section{Prototype}
\begin{frame}{Prototype}
    \begin{center}
      \begin{columns}[T] % The "T" option aligns column content to the top
        \begin{column}{.48\textwidth} % Left column, 48% of the text block width
          \includegraphics[width=0.7\linewidth]{../thesis-doc/images/prototype/prototype_on_head_visual.png} % Left image, full column width
        \end{column}
        
        \begin{column}{.48\textwidth} % Right column, 48% of the text block width
          \includegraphics[width=0.75\linewidth]{../thesis-doc/images/prototype/Lorenz.png} % Right image, full column width
        \end{column}
      \end{columns}
    \end{center}
\end{frame}

\begin{frame}{Prototype}
    \begin{center}
        \includegraphics[width=0.6\linewidth]{../thesis-doc/images/prototype/PrototypeConnection.png} % Right image, full column width
    \end{center}
\end{frame}

\begin{frame}{Prototype}
    \begin{center}
        \includegraphics[width=0.6\linewidth]{../thesis-doc/images/prototype/PCB_Description.png} % Right image, full column width
    \end{center}
\end{frame}

\begin{frame}{Prototype}
\begin{columns}[T] % The "T" option aligns column content to the top
    \begin{column}{.48\textwidth} % Left column, 48% of the text block width
      \includegraphics[width=0.9\linewidth]{../thesis-doc/images/prototype/flex_pcb_design_finding.png} % Right image, full column width
    \end{column}

    \begin{column}{.48\textwidth} % Right column, 48% of the text block width
      \includegraphics[width=1.05\linewidth]{../thesis-doc/images/prototype/pcb/flex_pcb_3D.png} % Right image, full column width
    \end{column}
  \end{columns}
    
\end{frame}

\begin{frame}{Prototype}
  \begin{columns}[T] % The "T" option aligns column content to the top
    \begin{column}{.31\textwidth} % Left column, 48% of the text block width
      \includegraphics[width=0.9\linewidth]{../thesis-doc/images/prototype/Earpod_Side1.png} % Left image, full column width
    \end{column}
    
    \begin{column}{.31\textwidth} % Right column, 48% of the text block width
      \includegraphics[width=0.9\linewidth]{../thesis-doc/images/prototype/Earpod_Front_2.pdf} % Right image, full column width
    \end{column}

    \begin{column}{.31\textwidth} % Right column, 48% of the text block width
      \includegraphics[width=0.9\linewidth]{../thesis-doc/images/prototype/Earpod_Side2.png} % Right image, full column width
    \end{column}
  \end{columns}
\end{frame}

\begin{frame}{Prototype}
  \begin{columns}[T] % The "T" option aligns column content to the top
    \begin{column}{.31\textwidth} % Left column, 48% of the text block width
      \includegraphics[width=0.9\linewidth]{../thesis-doc/images/prototype/Earpod_Side1_visual_markers.pdf} % Left image, full column width
    \end{column}
    
    \begin{column}{.31\textwidth} % Right column, 48% of the text block width
      \includegraphics[width=0.9\linewidth]{../thesis-doc/images/prototype/Earpod_Front_visual_markers.pdf} % Right image, full column width
    \end{column}

    \begin{column}{.31\textwidth} % Right column, 48% of the text block width
      \includegraphics[width=0.9\linewidth]{../thesis-doc/images/prototype/Earpod_Side2_visual_markers.pdf} % Right image, full column width
    \end{column}
  \end{columns}
\end{frame}

\section{Study 1}
\begin{frame}{Study 1: Localized Ear Temperature Measurement Study Procedure: Baseline Surveys and Environmental Influences}
\textbf{Hypothesis:}
\begin{overprint}
  \onslide<1-3>
    \begin{itemize}
        \item<1-> The temperature measured by sensors located behind the ear is lower compared to the other locations.
        \item<2-> The variance in temperature readings differs between indoor and outdoor settings.
        % \item<3-> Relative changes in temperature readings across different sensor locations will be interrelated.
        \item<3-> The temperature at the tympanic membrane has the greatest stability compared to other sensor locations.
        % \item<5-> Subject movement leads to significant changes in the temperature readings across various sensor locations.
    \end{itemize}
    \end{overprint}
    
    \begin{center}
        \includegraphics<1->[width=0.85\linewidth]{../thesis-doc/images/study1/Procedure_short.pdf} % Image visible from slide 1 onward
    \end{center}
\end{frame}

\begin{frame}
    \begin{center}
        \includegraphics[width=0.9\linewidth]{../thesis-doc/images/study1/Logging_person_10_0smoothed_raw_data.png} % Right image, full column width
    \end{center}
\end{frame}

\begin{frame}{Evaluation Study 1: Hypothesis 1}
    \begin{itemize}
        \item additional info or pictures
    \end{itemize}
\end{frame}

\begin{frame}{Evaluation Study 1: Hypothesis 2}
    \begin{itemize}
        \item additional info or pictures
    \end{itemize}
\end{frame}

% \begin{frame}{Evaluation Study 1: Hypothesis 3}
%     \begin{itemize}
%         \item additional info or pictures
%     \end{itemize}
% \end{frame}

\begin{frame}{Evaluation Study 1: Hypothesis 4}
    \begin{itemize}
        \item additional info or pictures
    \end{itemize}
\end{frame}

% \begin{frame}{Evaluation Study 1: Hypothesis 5}
%     \begin{itemize}
%         \item additional info or pictures
%     \end{itemize}
% \end{frame}

\section{Study 2}
\begin{frame}{Study 2: Study Course Under Stress Conditions: Impact on Temperature Measurements With Ear-Based Sensors}
\textbf{Hypothesis:}
\begin{overprint}
  \onslide<1->
  \begin{itemize}
    \item<1-> A measurable rise in temperature occurs during stress-inducing activities.
    \item<2-> There will be variability in temperature changes across different types of stress tests.
  \end{itemize}
\end{overprint}

\begin{center}
  \includegraphics<1->[width=0.95\linewidth]{../thesis-doc/images/study2/Procedure2_short.pdf} % Image visible from slide 1 onward
\end{center}
\end{frame}

\begin{frame}
    \begin{center}
        \includegraphics[width=0.9\linewidth]{../thesis-doc/images/study2/p04/raw_hrv_data_participant_4.png} 
    \end{center}
\end{frame}

\begin{frame}
    \begin{center}
        \includegraphics[width=0.9\linewidth]{../thesis-doc/images/study2/p04/Logging_study2_p04_0smoothed_raw_data.png} 
    \end{center}
\end{frame}

% \begin{frame}{Evaluation Study 2: Hypothesis 1}
%     \begin{itemize}
%         \item additional info or pictures
%     \end{itemize}
% \end{frame}

% \begin{frame}{Evaluation Study 2: Hypothesis 2}
%     \begin{itemize}
%         \item additional info or pictures
%     \end{itemize}
% \end{frame}

\section{Conclusion \& Future Work}
\begin{frame}{Conclusion}
    \begin{itemize}
        \item additional info or pictures
    \end{itemize}
\end{frame}

\begin{frame}{Future Work}
    \begin{itemize}
        \item TSST for Stress Detection with Increased Sample Size
        \item Detection of Circadian Rhythm
        \item Early Detection of Disease
        \item Cycle Tracking for Women
        % \item Real-World Applications
        % \item Approaches to Machine Learning
    \end{itemize}
\end{frame}

\appendix
\beginbackup


\begin{frame}{Literature}
    \printbibliography
\end{frame}

\section{Study Descriptions}
\begin{frame}
    \begin{center}
        \includegraphics[width=0.9\linewidth]{../thesis-doc/images/study1/Procedure.pdf} % Image visible from slide 1 onward
    \end{center}
\end{frame}

\begin{frame}
    \begin{center}
        \includegraphics[width=0.9\linewidth]{../thesis-doc/images/study2/Procedure2.pdf} % Image visible from slide 1 onward
    \end{center}
\end{frame}

\section{Implementation}
% \begin{frame}{Implementation}
%     \includegraphics[width=0.45\linewidth]{../thesis-doc/images/ArduinoCodeProcedure.pdf} % Right image, full column width
% \end{frame}

\begin{frame}{Implementation}
    \begin{itemize}
        \item Arduino code
        \begin{itemize}
            \item setup() for initialization of temperature and imu sensors and sd logger
            \item loop() for measuring data (temperature sensors with $8,3Hz$, imu data with $50Hz$)
        \end{itemize}
        \item Button action
        \begin{itemize}
            \item Single click: next phase
            \item Double click: stop measurement
        \end{itemize}
        \item Store data on SD-card
        \begin{itemize}
            \item 6 temperature sensors (object and sensor temperature)
            \item IMU data
            \item Timestamp
            \item ID (phase)
        \end{itemize}
    \end{itemize}
\end{frame}

\begin{frame}{Implementation}
    \includegraphics[width=\linewidth]{../thesis-doc/images/prototype/MeasurementRawDataSnippet_short.png} % Right image, full column width
\end{frame}

\backupend

\end{document}