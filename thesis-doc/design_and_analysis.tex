%% entwurf.tex
%% $Id: entwurf.tex 28 2007-01-18 16:31:32Z bless $
%%
%% ==============================
\chapter{Design and Analysis}
\label{ch:Design}
This master's thesis aims to compare different positions on the ear for measuring body temperature. 
For this purpose, a prototype was developed that allows temperature measurements at different positions in and around the ear. 
Studies were then conducted to collect temperature data at different locations in and around the ear, then temperature changes under stress were observed.
The data collected from both studies were then analyzed and evaluated.
This chapter first focuses on the OpenEarable platform, which is the cornerstone of the prototype. 
Next, the sensors critical to temperature measurement are discussed.
After laying the groundwork for understanding the prototype, the development and approach to designing the prototype are explained.
In addition, the methodology and process of the two studies are described in detail.
The results of the studies are presented in Chapter \ref{ch:Evaluation}.

\section{Platform: OpenEarable}
\label{ch:Design:Prototype:OpenEarable}
% Das OpenEarable ist ein vom TECO Institut eintworfener Prototyp, mit welchem es möglich ist, ohrbasierte Messungen durchzuführen. 
In Chapter \ref{Background:SensingWithEarables:OpenEarable}, the OpenEarable platform is thoroughly introduced, emphasizing its interaction with other components. 
The OpenEarable was designed for rapid prototyping and exploration of novel earable applications.
This thesis successfully utilized it as the foundation for the prototype. 
Figure \ref{fig:design:prototype_connection} illustrates the seamless collaboration between all components. 
The OpenEarable uses an Arduino Nano 33 BLE and enables the use of Arduino-based software. It connects to the PCB and FlexPCB via a 4-pin connector. 
The developed software collects data from the temperature sensors and effectively utilizes the resources of the OpenEarable. 
In addition, the IMU data already available on the OpenEarable is read out.
The study data is permanently stored on the SD card and the termination of the study is triggered by a double click on a push button.
In summary, the OpenEarable platform proved to be a versatile and effective foundation for the prototype, allowing seamless interaction between its components and facilitating data collection from the temperature sensors. 
By leveraging Arduino-based features and user-friendly design, OpenEarable was successfully used in this study to explore novel earable applications and conduct the user study easily and efficiently.

% TODO: PCB Schematic in den Anhang vermutlich packen, sehr wichtig!!!

\section{Sensors}
\label{ch:Design:Prototype:Sensors}

Based on research and experience at the TECO Institute, the MLX90632 sensor was selected for its suitability for the project.

The MLX90632 sensor is an infrared temperature sensor known for its high accuracy in temperature measurements. This enables reliable applications where precise temperature tracking is needed.
Furthermore, non-contact temperature measurement is a huge advantage. 
The MLX90632 can measure temperature without physical contact with the object or body part, providing a non-invasive and convenient ear temperature monitoring option.
In order to place the temperature sensors in the necessary locations to measure temperature, the sensors must be appropriately small. 
Since the MLX90632 has a small form factor (3x3mm), this is optimal for the application needed.
Additionally, the MLX90632 is readily available on the market and has existing Arduino libraries. 
This availability and compatibility with Arduino simplify the integration process and save valuable development time.
In addition, the sensor is designed for low power consumption, making it suitable for battery-powered devices. 
Given the small battery size in the developed prototype, low power consumption is critical for extended operation during the study.
The MLX90632 offers fast response times and allows for real-time temperature monitoring and fast updates. 
Some difficulties arose when writing the EEPROM, so the default value was left at 2Hz.
The measurement rate was adjusted due to implementation details of the library, because the library waits the time until a new measurement value arrives.
Since this failed, the library was modified.
This process is described in more detail in Chapter \ref{ch:Implementation}.
The high sensitivity to temperature changes allows the MLX90632 to accurately detect even minor variations. 
This sensitivity is advantageous for precise temperature tracking.
In addition, the sensor has applications in both the consumer and industrial markets due to its accuracy and reliability. 
This versatility makes it an excellent choice for this master's thesis project, which involves working with an Arduino Nano 33 BLE.

The MLX90632 has 4 pins that must be connected. 
Beside 3.3V and Ground the MLX90632 has a SCL and SDA connector. 
SCL stands for the clock signal and SDA for the data flow.

Overall, infrared temperature sensing capabilities, accuracy, easy availability, non-invasive measurement, compact size and compatibility with Arduino make the MLX90632 an ideal choice for developing an ear temperature monitoring system.

% \subsection{Calibration Mechanism for Sensor Fusion}
% \label{ch:Design:Prototype:Sensors:Calibration}
% In the context of this study, the primary focus was on investigating the variability of temperature measurements at diverse positions in and around the ear, rather than obtaining a universally accurate temperature value. 
% The sensors used had a factory accuracy of \( \pm0.2^\circ \text{C} \), but showed variability of up to \(1.1^\circ \text{C}\) under controlled conditions in preliminary self-tests. 
% This motivated the implementation of a calibration mechanism to harmonize the sensor data. 
% Different reference temperatures ($25, 30, 35, 40, 45^\circ$ C) and fitting models, varying from constant and linear offsets to polynomial offsets of different degrees (2, 4, 8, 16, 32), were used for calibration. 
% The sensors were placed at uniform distances from a tempered metal plate to quantify systematic differences. 
% Despite the controlled experimental setup, significant discrepancies were found between sensor readings, highlighting the need for an internal calibration mechanism to improve data consistency across multiple measurement locations.

% Data logging was performed for calibration. 
% The sensors, before being installed in the prototype, were arranged so that they were all positioned on a metal surface and aligned at an equal distance from another metal plate.
% This was to take advantage of the thermal properties of the metal to obtain consistent temperature measurements across all 6 sensors.
% The target metal plate was set to different reference temperatures ($25, 30, 35, 40, 45^\circ$ C) and a series of measurements was started.
% Then, the recorded measurement data were concatenated and different offsets, including constant, linear and polynomial offset with the degrees (2, 4, 8, 16, 32), were calculated.

% The MLX90632 sensor uses a special formula to convert measured values to temperature specifications, which is described in Chapter \ref{ch:Design:Prototype:Sensors}.
% Special attention is paid to the emissivity factor, a dimensionless quantity between 0 and 1, which describes the ratio of the energy radiated from the surface to the energy of an ideal black body.
% For measurements of body temperature, this factor must be set to 0.98.
% Although the calibration was performed on metal, which requires an emissivity factor of 1, the factor was left at 0.98 to adapt the calibration to the human body.
% The heated platform of a 3D printer served as the heat source for the metal plate, as it could cover the required temperature ranges.
% The factors for the respective offsets were then saved in a JSON file, which can be used during analysis to calibrate the recorded data.

\section{Prototype}
\label{ch:Design:Prototype}
To measure the temperature as planned in Section \ref{ch:Introduction:PlannedApproach}, a custom-built prototype was developed as part of this master's thesis.
The prototype consists of two components: an earpiece that resembles an in-ear headphone and a component placed behind the ear that resembles a hearing aid. TECO's OpenEarable platform for ear-based observations is integrated into the behind-the-ear component. This component serves as the central interface and houses the Arduino on which all code is executed. Additionally, a circuit board with three temperature sensors is connected to measure the temperature behind the ear.
The second component is placed in the ear and is also controlled by the OpenEarable via the Arduino Nano33 BLE installed there. The OpenEarable acts as the basis for reading and storing data from the sensors connected via I2C. The relationship and interaction of the components are visually represented in Figure \ref{fig:design:prototype_connection}.
To ensure functionality and protection of the hardware, custom 3D-printed enclosures were created for both the behind-the-ear component and the in-the-ear component. 
Figure \ref{fig:design:prototype_on_head_visual} shows the final result, worn by a participant and also visually demonstrates the positioning of the components.
As shown in Figure \ref{fig:design:prototype_on_head_visual}, the custom cases add durability and a high-end appearance to the product, improving its overall usability and aesthetics.
The use of the 8-pin cable effectively connects the two components, additionally ensuring good comfort and fit.

\begin{figure}[t]
    \centering
    \includegraphics[width=0.48\textwidth]{thesis-doc/images/prototype/prototype_on_head_visual.png}
    \includegraphics[width=0.48\textwidth]{thesis-doc/images/prototype/Lorenz.png}
    \caption{Sketched view while carrying the components, next to it a real view. The PCB behind the ear measures in three positions (bottom, middle, top) and is connected via an 8-pin connector to the FlexPCB, which is already wrapped in the case here in the right picture. In the ear is the second component, which is used to measure at the concha, the ear canal and the tympanic membrane. The wiring of the two components results in good stability when worn, so that the device cannot fall off. }
    \label{fig:design:prototype_on_head_visual}
\end{figure}

\begin{figure}
    \centering
    \includegraphics[width=\textwidth]{thesis-doc/images/prototype/PrototypeConnection.png}
    \caption{Visual representation of the prototype and the interaction of all components. The PCB is connected to the OpenEarable v1.3 with a 4-pin connector. In the OpenEarable is an Arduino Nano33 BLE, with which it is possible to control the multiplexer (TCA9548A) via I2C. Through this, every sensor value on the PCB and also on the FlexPCB can be read out, since the FlexPCB is also connected to the multiplexer via the 8-pin connector.}
    \label{fig:design:prototype_connection}
\end{figure}

\begin{figure}
    \centering
    \includegraphics[width=\textwidth]{thesis-doc/images/prototype/PCB_Description.png}
    \caption{Representation of the front and back of the PCB. The multiplexer can be seen on the front, which is controlled by the OpenEarable via the 4-pin connector. The three other MLX90632 are then connected by the FlexPCB via the 8-pin connector. In addition, an LED is connected to the front, which lights up green if no short circuit is generated. The temperature sensors can be seen on the back, but only three of the five connections visible in the design are used.}
    \label{fig:design:pcb_description}
\end{figure}

\subsection{Temperature Measurements Behind the Ear}
\label{ch:Design:Prototype:BehindEar}

The temperature behind the ear is measured at three positions, as can be seen in Figure \ref{fig:design:pcb_description} on the back of the PCB and conceptually in Figure \ref{fig:ear_measurement_positions}.
To position the temperature sensors at the locations chosen in Section \ref{ch:Introduction:PlannedApproach}, a PCB was developed that has the sensors installed at the appropriate locations. 
The PCB has been designed so that only the temperature sensors are placed on the back. This allows the PCB to be placed entirely in the bottom of the case, while the temperature sensors peek out through matching openings in the case. On the front of the PCB are all the other components, including a 4-pin connector and an 8-pin connector.
The 4-pin connector is used to connect to the OpenEarable. This connection allows the OpenEarable to communicate with the PCB via I2C, as the OpenEarable also has a special 4-pin connector for exactly such a purpose.
To be able to control the second component (the earpiece) via I2C later on as well, the 8-pin connector was added to establish a connection to the second component.
Via I2C, the built-in multiplexer is addressed, with which one of the eight possible applied lines can be switched and read out. The eight possible through-connections of the multiplexer are connected to all temperature sensors, including those of the FlexPCB via the 8-pin connector.
In addition, an LED is installed on the PCB to directly indicate a possible short circuit.
For each temperature sensor, the signals Ground, Power (3.3V) as well as SCL (clock signal) and SDA (data transmission) are required, as shown in Figure \ref{fig:design:pcb_description}. Communication with the multiplexer can be handled through the 4-pin connector.
To connect the three temperature sensors of the FlexPCB, a total of 12 signals are required, which can be reduced somewhat. For this purpose, the ground and power signals can be used together, resulting in a total of 8 signals being transmitted.

A case has now been developed around the OpenEarable and the custom-made PCB to enable a comfortable fit.
Above the PCB, the battery is placed in the enclosure so that no long cables are needed for the power supply to record the data in the study conducted.
The OpenEarable is placed above this.
The dimensions of the PCB are exactly the same as the OpenEarable to keep the case as small and compact as possible.
The sensor used requires an angle of $ 50 ^ \circ$ around itself for the temperature to be reliably measured. 
This was taken into account.
% The layered representation for visualization can be seen in Figure \ref{fig:design:prototype_behind_head_layered_view}.

% \begin{figure}
%     \centering
%     \includegraphics[width=\textwidth]{thesis-doc/images/prototype/Prototype_PCB_layered_view.png}
%     \caption{Layered view of the component behind the ear. At the bottom is the designed PCB as the temperature sensors can look out towards the skin. The battery is placed between the PCB and the OpenEarable.}
%     \label{fig:design:prototype_behind_head_layered_view}
% \end{figure}

\subsection{Temperature Measurements in the Ear}
\label{ch:Design:Prototype:Earpiece}

The second component now enables temperature measurement in the ear area. The FlexPCB itself is only equipped with components on the front side. Thereby, the 8-pin connector that connects the FlexPCB to the PCB is located, as described in Section \ref{ch:Design:Prototype:BehindEar}.
Additionally, three temperature sensors are placed on the FlexPCB to sense the positions in the ear and Concha described in Section \ref{ch:Introduction:PlannedApproach} and Figure \ref{fig:ear_measurement_positions}.
The FlexPCB was designed to extend through the component. 
On the one hand, the 8-pin connector extends outward to connect to the PCB. 
On the other hand, the PCB extends along the outside of the case to the earbud, allowing the FlexPCB to snake through. 
The tip of the FlexPCB also contains a temperature sensor mounted in the earplug and aimed directly at the tympanic membrane. 
Another temperature sensor is aimed at the ear canal and is located on the outer edge of the earplug. The third temperature sensor is aimed at the concha.
The component was modeled on the design of an AirPod, but heavily modified afterward. 
The original AirPods design is freely available on TinkerCAD and was used as the basis for the component shape. 
The inside of the design was completely hollowed out to allow cables to be routed through the case. Additionally, an adapter was added to the side to fit the redesigned earpod. 
An earbud can be attached here to ensure that the earbud penetrates further into the ear than usual compared to conventional in-ear headphones. 
This enables precise temperature measurement in the direction of the tympanic membrane. 
A temperature sensor is attached to the tip of the earbud to perform basic temperature measurements.
The three temperature sensors on the FlexPCB can be switched via the multiplexer that is connected to the PCB. 
This allows for precise selection and acquisition of the desired measurements. Figure \ref{fig:design:prototype_earpiece_views} shows a sketch and other images of the final component.

\begin{figure}[!h]
    \centering
    \includegraphics[width=0.48\textwidth]{thesis-doc/images/prototype/flex_pcb_design_finding.png}
    \includegraphics[width=0.48\textwidth]{thesis-doc/images/prototype/Earpod_Front.png}
    \includegraphics[width=0.48\textwidth]{thesis-doc/images/prototype/Earpod_Side1.png}
    \includegraphics[width=0.48\textwidth]{thesis-doc/images/prototype/Earpod_Side2.png}
    \caption{Sketch and real view of the earpiece from different perspectives.
The sketch shows the first idea of how the FlexPCB has to be designed so that it gets through the earplug to the tip to measure in the direction of the tympanic membrane. The real views provide an impression of how the component is composed. On the bottom right picture, you can see the sensors on the concha and ear canal, on the top right picture the sensor directing to the tympanic membrane. The bottom left picture displays the 8-pin connector.}
    \label{fig:design:prototype_earpiece_views}
\end{figure}

\section{Study}
\label{ch:Design:Study}
In this master's thesis, the temperature is to be measured at different positions of the ear in order to enable temperature measurement over a longer period of time. 
For this purpose, a study is to be carried out in which the temperature is measured at different positions and then compared. 
For this purpose, a prototype has been developed, which has been described in detail in the previous chapters. 
With this prototype it is possible to measure the temperature at three positions behind the ear, at the auricle, in the ear canal and on the eardrum.
The goal is to compare the different positions after recording the data. 
Within the context of this master's thesis and related studies, several hypotheses arise that have the potential to clarify important aspects of ear temperature measurement.
To this end, two studies were designed to provide evidence for the hypotheses previously described in Section \ref{ch:Introduction:PlannedApproach}.
In addition, the issue of reproducibility and consistency of measurements is of interest. 
The hypothesis is that the measurement results will remain the same in different subjects despite differences in human physiology. 
The importance of the initial acclimation time of the sensors is also considered. 
Here, it is assumed that the 20-minute acclimation period is sufficient to ensure a stable temperature measurement {cite{chagllae.MeasurementCoreBody2018}}. 
The role of environmental variables is also considered. 
The results can be used to define further research questions and applications in the field of temperature measurement by wearables.

\subsection{Study 1: Localized Ear Temperature Measurement Study Procedure: Baseline Surveys and Environmental Influences}
\label{ch:Design:Study:Study1}
Two studies are conducted as part of this master's thesis. 
The first study deals with the comparison of temperatures at the different sites.
For this purpose, a study is designed to test the different hypotheses and to provide the best possible data basis. 
The study begins by giving the subject an introduction to the study. 
After the subject has signed the privacy statement, as well as the informed consent form, a temperature measurement of the right ear is taken with a thermometer (BRAUN ThermoScan 7) to obtain a reference value for the temperature in the ear. 
This measurement is taken again at the end of the study.
The prototype is then attached to the subject's ear. 
This phase is critical because the correct positioning of the sensors is crucial for the quality of the recorded data.
Since the component behind the ear tends to protrude a bit from the skin after a while, it is taped behind the ear to ensure that it is securely in place and would not slip even during light activity. 
This is observed during preliminary testing. 
Care is also taken to ensure that this did not affect the temperature measurement of the sensors.
This also includes external influences, such as wind, sun or other.
To ensure that the prototype's battery would last throughout the entire period, a power bank is attached, as initial results suggested that battery life is between one and two hours.
After the prototype is installed, an acclimation phase of 20 minutes follows, during which data is already being recorded. 
The test subject is now sitting alone in a chair in a room at the TECO Institute.
This phase is to ensure that the sensors have enough time to adapt to the physiological conditions of the test subject and to allow stable measurements.
After the acclimatization phase is completed, the study investigator entered the room and pressed the prototype button to classify the next phase in the data as well.
Subsequently, the subject is again alone in the room. 
The subject spends another 20 minutes in a seated position in a room at the institute where all other subjects have also conducted the study. 
The room is not air-conditioned and the windows are closed before the measurement.
In addition, the room temperature and humidity are noted between phases and at the beginning and end of the measurement. 
Subsequently, the subject answers the questionnaire and thus ends the study.
The purpose of this phase is to establish a baseline for the temperature measurements and to check the consistency of the sensors under stable conditions.
The next step is to investigate the influence of environmental variables. 
Subjects will be asked to walk outdoors for 20 minutes. 
This is done to analyze the response of the sensors to sudden changes in temperature and environment and to evaluate the adaptability of the system.
After returning to the room, the subjects take their seats again in their original places and remain in a seated position for another 20 minutes. 
This phase allowed for quantification and analysis of variations in sensor readings due to being outdoors.
Throughout the time in the room, subjects are allowed to watch animal documentaries that elicited no or minimal anxiety or happiness. 
Subsequently, subjects still answer a questionnaire that answers important information such as demographics, health, current weather conditions, and prototype comfort.
After answering the questionnaire, the study is finished for the subject.
A total of 12 subjects will be used in the study to provide a sufficient database for statistical analysis.
A visual diagram of the procedure of Study 1 can be seen in Figure \ref{fig:design:study1:procedure}.

\begin{figure}[t]
    \centering
    \includegraphics[width=\textwidth]{thesis-doc/images/study1/Procedure.pdf}
    \caption{Figure illustrating the procedure of Study 1, consisting of four 20-minute phases: acclimatization, sitting, outdoor walking, and relaxation. Ground truth temperature is recorded using a BRAUN Thermoscan 7 thermometer before and after the study.}
    \label{fig:design:study1:procedure}
\end{figure}

Through the design it should be possible to analyze the established hypotheses.
The baseline should provide a general feeling for the temperature measurements at the different positions.
In addition, through Phase 2, it is hoped to better assess the temperature differences during Phase 3 (outdoor phase).
In the relaxation phase (Phase 4), it should be observed how the sensors settle back to their normal level.

For Hypothesis 1, that the sensors behind the ear measure lower temperatures than the sensors at the tympanic membrane, the ear canal and the concha, this design of the study is an ideal prerequisite, because besides the baseline also external influences of the outdoor phase can be considered.
Also for Hypothesis 2, which considers the variance differences between the controlled and uncontrolled environment, the study design provides a good basis, because both phases are covered.
For Hypothesis 3, which tests whether certain sensors behave more or less similarly, the study design provides a good basis.
In the outdoor phase (Phase 3), the correlation is expected to be very high, as the test subjects move into a colder environment and thus the correlation between the sensors is expected to be very high.
Likewise, Hypothesis 4 can be perfectly considered and evaluated. 
Here it has to be shown that the temperature of the sensor directed to the tympanic membrane is the most stable. 
It is expected that the subjects are exposed to external conditions in the outdoor phase (Phase 3) and a clear difference can be seen especially in this phase.
Since the IMU data are always recorded in parallel, Hypothesis 5 can also be analyzed well with the study design.
Here, it is examined whether increased human movements lead to temperature changes. 
However, a distribution is not directly clear. 
Since the movement is clearly higher in the outdoor phase, the temperature change can be tested here, but this can also be caused by external environmental influences and not only by the pure movement.

However, a further subdivision into two phases (movement and walking oudoors) is beyond the scope of the study.
Through this carefully designed procedure, the study should help test the formulated hypotheses and provide valuable insight into the performance of the developed prototype and its potential applications.

\subsection{Study 2: Study Course Under Stress Conditions: Impact on Temperature Measurements With Ear-Based Sensors}
\label{ch:Design:Study:Study2}
In the second study, the influence of stress-induced physiological changes on temperature measurements at different locations of the ear is investigated. 
The study begins similarly to the first study in that the subject is given an introduction at the beginning and the explanation of the study procedure.
After the subject signs the privacy statement and informed consent, a thermometer (BRAUN Thermoscan 7) is used to measure the temperature in the same ear (right) where the prototype will be placed.
The prototype is attached as in Study 1.
In Study 2 the attachment of the reference meter is added.
A HRV chest strap (Polar H9) will provide heart rate variability as ground truth for this purpose.
The chest strap is connected to the smartphone using the "EliteHRV" app and the measurement is started simultaneously with the prototype.
The app is used in such a way that the subject cannot see any HRV readings, only a time indicating the length of the current measurement.
An initial 20-minute acclimatization phase, during which the sensors and the subjects adapt to the environmental conditions, is followed by a 15-minute measurement phase in a sitting position.
Between the phases, the person conducting the study enters the room and presses the button of the prototype in order to be able to distinguish the individual phases in the data as well.
After that, the subjects are exposed to a stress situation. 
\begin{figure}[!t]
    \centering
    \includegraphics[width=\textwidth]{thesis-doc/images/study2/Procedure2.pdf}
    \caption{Figure outlining Study 2, which measures temperature changes under stress. The procedure includes a 20-minute acclimatization, 15 minutes of sitting, three stress tests, and a 15-minute relaxation phase. Ground truth is obtained using a BRAUN Thermoscan 7 thermometer before and after the study.}
    \label{fig:design:study2:procedure}
\end{figure}
At the beginning of the stress-induced phase, the subjects solve a Stroop test, then an N-Back test and finally a mathematics test. 
Here, the subject has the task of noting the time at which he starts the respective stress test.
The time is taken from the "EliteHRV" app.
The Stroop test tests the subject's attention by alternately displaying words in different colors. 
The subject is asked to interpret and select the correct color from a range of colors.
To do this, the subject has a fixed number of words on the basis of which they must determine the color. 
The time in which the subject processes this number is measured.
The words themselves are always color words, but the words do not always match the color in which they are displayed.
When the word and the color do not match, reaction time and the number of errors increase. 
This creates an initial stress situation for the subject \cite{StroopCompetitionSocialEvaluative}. 
Next, an N-back test with 2 dimensions is performed. 
Here, the subject is told a letter via a voice output and shown a position in a tic-tac-toe box. 
The subject must now recall N steps and indicate whether the letter or position has already been named or shown in the last N iterations. 
This test is performed with $N={1,2,3}$ and is designed to elicit strain and stress \cite{liangEffectAcuteStress2023}.
Last, the subject is presented with a mathematical test. 
This involves several mathematical tasks in which 4 answer options are always available.
Here, the subject has 8 minutes to solve as many tasks as possible. 
This is also designed to trigger a stress situation \cite{caviolaStressTimePressure2017}.

All three tests are designed to trigger stressful situations and challenge the subject in a different way each time.
For the Stroop test, there is no visual timer, but the number of words is limited. 
There is a high score for the test, which is intended to motivate the subject to perform as well as possible. 
In the N-Back test, the time is strictly predetermined, and the subject has the opportunity to answer within three seconds.
Here, the subject is at risk of becoming frustrated if he or she fails to complete one or more sub-steps.
The mathematical test has a time limit of 8 minutes, in which the subject must solve as many tasks as possible.
Again, the subject is put under pressure in a different way that is intended to trigger stressful situations.
This scenario is chosen because a large study is not possible due to time constraints. 
The scientific standard for induction in standardized stress induction tests is the Trier Social Stress Test (TSST). 
However, the chosen test covers the requirements and should allow to induce stress in the subjects.
To check the stress level of the subjects, heart rate variability (HRV) is measured in addition to the temperature data. 
These markers provide a solid scientific basis for evaluating stress induction and its effects on the measured temperature values.
Here, the HRV values provide the ground truth in stress classification.
The stress induction phase is followed by a further 15-minute measurement phase in a seated position, during which the subjects are not exposed to any other stressful situations. 
This serves to record the recovery processes and their effects on the measured parameters. 
The procedure is visually depicted in Figure \ref{fig:design:study2:procedure}.
The study is used to analyze several hypotheses, which have already been described in section \ref{ch:Introduction:PlannedApproach}.
Hypothesis 1 states that temperature will increase under stress.
By recording the baseline in Phase 2 and the stress-induced phase in Phase 3, this can be solidly tested.
Hypothesis 2 deals with the different temperature differences between the stress tests.
Since a total of three stress tests are performed, this can also be tested.

In Study 2, the effect of different stress-inducing tests - the Stroop test, the N-back test and a timed mathematical test - on ear temperature measurement is investigated.
At the same time, heart rate will be measured as an additional physiological marker of stress. 
The study includes an initial acclimation phase a pre-stress measurement phase, a stress induction phase, and a post-stress measurement phase to comprehensively assess changes in ear temperature under stress conditions.