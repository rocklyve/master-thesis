\thispagestyle{empty}
\vspace*{42\baselineskip}
\hbox to \textwidth{\hrulefill}
\par
Ich versichere wahrheitsgemäß, die Arbeit selbstständig angefertigt, alle benutzten Hilfsmittel vollständig und genau angegeben und alles kenntlich gemacht zu haben, was aus Arbeiten anderer unverändert oder mit Abänderungen entnommen wurde.

Karlsruhe, den 01.10.2023

\cleardoublepage

\vspace*{1em}
\begin{center}
	\textbf{Zusammenfassung}
\end{center}
\par
Diese Masterarbeit untersucht die Rolle von Ohrbasierten Temperaturmessungen und die optimale Platzierung von Sensoren für tragbare Anwendungen.
Ein Prototyp wurde entwickelt, um Temperaturdaten an verschiedenen Positionen des Ohrs zu sammeln.
Zwei Studien wurden durchgeführt, um unterschiedliche Hypothesen zu testen.
Die erste Studie fokussierte sich auf die lokale Temperaturmessung am Ohr und ermittelte, dass die Sensoren konsistente und verlässliche Daten lieferten.
Die zweite Studie konzentrierte sich auf den Einfluss von Stress auf die Ohrtemperatur.
Es wurde festgestellt, dass Stress messbare Temperaturänderungen verursachte, was durch zusätzliche Herzfrequenzmessungen bestätigt wurde.
Diese Erkenntnisse sind besonders relevant für die Entwicklung von Wearables, die Stress oder andere physiologische Zustände überwachen könnten.
Die Arbeit schließt erfolgreich die Lücke zwischen Theorie und Praxis und liefert eine solide Grundlage für zukünftige Forschungen, einschließlich der Erkennung von zirkadianen Rhythmen und der Zyklusverfolgung für Frauen.

\cleardoublepage
\vspace*{1em}
\begin{center}
	\textbf{Abstract}
\end{center}
\par
This master's thesis investigates the role of ear-based temperature measurements and optimal sensor placement for wearable applications.
A prototype was engineered to capture temperature data at various ear positions.
Two studies were conducted to test different hypotheses.
The first study focused on localized ear temperature measurements and found that the sensors provided consistent and reliable data.
The second study examined the impact of stress on ear temperature and revealed that stress led to measurable changes in temperature, further corroborated by additional heart rate measurements.
These findings are especially pertinent for the development of wearables that could monitor stress or other physiological states.
The thesis successfully bridges the gap between theoretical concepts and practical implementation, providing a solid foundation for future research, including the detection of circadian rhythms and cycle tracking for women.

\cleardoublepage

