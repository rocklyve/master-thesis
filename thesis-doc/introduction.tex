%% Einleitung.tex
%% $Id: einleitung.tex 28 2007-01-18 16:31:32Z bless $
%%

\chapter{Introduction}
\label{ch:Introduction}

\section{Motivation}
With the increasing interest in wearable health monitoring devices, temperature measurements have become a critical aspect of biometric data collection. The ear canal is a promising location for body temperature measurement because it provides a stable and easily accessible location for measurements.

In addition, this location is less susceptible to body movement. However, the optimal sensor placement and measurement methodology for ear-based temperature monitoring is still an open research question.

\section{Problem}
Previous studies have reported discrepancies in ear-based temperature measurements due to factors such as sensor location, skin contact, and calibration. In addition, the potential impact of different measurement positions on the accuracy and reliability of ear-based temperature measurements is still unclear.

\section{Question}
In this thesis, we will examine temperature sensors, each placed at different locations in the ear canal to compare the sensor measurements to those of a medically certified thermometer, which will be the ground truth here. The research question is to find the best location for ear-based temperature measurement and the way it affects the accuracy and stability of those. Afterwards, the resulting optimal location will be taken as a basis to detect a temperature-dependent event of the body and evaluate whether it provides conclusive results.

\section{Planned approach}
We will use the OpenEarable platform to attach various temperature sensors to the ear canal. The sensors will be placed at various locations including the concha, inside the ear canal, on the eardrum, and behind the pinna on the mastoid. Participants will be asked to wear the sensors on one ear in a study under laboratory conditions. At the same time the participants core body temperature will be measured with a medical thermometer on the other ear. The resulting date will be collected and analyzed afterwards to determine the best sensor position for ear-based temperature measurement. This position will then be used to classify temperature-related events of the body.

\section{Expected results}
We expect that our study will provide insights into the optimal placement of sensors for temperature monitoring in or directly around the ear. We are looking forward to identify variations in temperature measurements at different positions in the ear canal. Our results will inform the development of more accurate and reliable wearable devices for biometric applications. In addition, the optimally determined position will be taken as a new basis to make analyses on various bodily temperature changing events.