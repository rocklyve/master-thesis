%% Beispiel-Präsentation mit LaTeX Beamer im KIT-Design
%% entsprechend den Gestaltungsrichtlinien vom 1. August 2020
%%
%% Siehe https://sdqweb.ipd.kit.edu/wiki/Dokumentvorlagen

%% Beispiel-Präsentation
\documentclass[en]{sdqbeamer} 
 
%% Titelbild
\titleimage{banner_2020_kit}

%% Gruppenlogo
\grouplogo{teco_logo.png} 

%% Gruppenname und Breite (Standard: 50 mm)
\groupname{
Chair of Pervasive Computing Systems/TECO\\
Institute of Telematics, Department of Informatics\\
}
%\groupnamewidth{50mm}

% Beginn der Präsentation

\title[Ear-Based Temperature Probing]{Ear-Based Temperature Probing: \\ Sensor Placement and Fusion for Wearable Applications}
\subtitle{Chair of Pervasive Computing Systems / TECO} 
\author[David Laubenstein]{David Laubenstein}

\date[05/10/2023]{May 10, 2023}

% Literatur 
 
\usepackage[citestyle=authoryear,bibstyle=numeric,hyperref,backend=biber]{biblatex}
\usepackage[inkscapeformat=png]{svg}
\addbibresource{presentation.bib}
\bibhang1em

\begin{document}
 
%Titelseite
\KITtitleframe

%Inhaltsverzeichnis
% sollte weg laut betreuern
% \begin{frame}{Inhaltsverzeichnis}
% \tableofcontents
% \end{frame}

\section{Motivation}
\begin{frame}{Motivation}
asdf
\end{frame}

\section{Problem}
\begin{frame}{Problem}
asdf
\end{frame}

\section{Question}
\begin{frame}{Question 1}
    \begin{itemize}
        \item asdf
        \item asdf
        \item asdf
        \item asdf
        \item asdf
    \end{itemize}
\end{frame}

\begin{frame}{Question 2}
asdf
\end{frame}

\section{Planned Approach}
\begin{frame}{Planned Approach}
asdf
\end{frame}

\section{Expected Result}
\begin{frame}{Expected Result}
asdf
\end{frame}

\appendix
\beginbackup

% TODO: Add frames for appendix

\begin{frame}{Literatur}
\begin{exampleblock}{Titelbilder Quelle:}
    asdf    
\end{exampleblock}
\end{frame}

\begin{frame}{Literatur}
    \printbibliography
\end{frame}

\section{Farben}
\backupend

\end{document}